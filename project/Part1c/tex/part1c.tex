\documentclass[aps,prl,groupedaddress]{revtex4-1}  % One column - tight
%\documentclass[aps,prl,preprint,groupedaddress]{revtex4-1}  % One column - spread
%\documentclass[aps,prl,reprint,groupedaddress]{revtex4-1}  % Two column
%\documentclass[aps,prl,preprint,superscriptaddress]{revtex4-1}
%\documentclass[aps,prl,reprint,groupedaddress]{revtex4-1}

% Use the \preprint command to place your local institutional report
% number in the upper righthand corner of the title page in preprint mode.
% Multiple \preprint commands are allowed.
% Use the 'preprintnumbers' class option to override journal defaults
% to display numbers if necessary
%\preprint{}

% You should use BibTeX and apsrev.bst for references
% Choosing a journal automatically selects the correct APS
% BibTeX style file (bst file), so only uncomment the line
% below if necessary.
%\bibliographystyle{apsrev4-1}

% % % % % % % % % % % % Noam% % % % % % % % % % % % 
% Added preamble commands:
\usepackage[utf8]{inputenc}
\usepackage{array}
\usepackage{mathrsfs}
\usepackage{multirow}
\usepackage{amsmath}
\usepackage{graphicx}
\graphicspath{{figures/}{figures/phase_diagram/}{figures/levels_scheme/}{figures/decomposition/}{figures/phase_diagram/}{figures/energies_transitions_ratios/}}
\usepackage[unicode=true,pdfusetitle,bookmarks=true,bookmarksnumbered=false,bookmarksopen=false,
 breaklinks=false,pdfborder={0 0 1},backref=false,colorlinks=false]{hyperref}
%%%%%%%%%%%%%%%%%%%%%%%%%%%%%% User specified LaTeX commands.
\usepackage{braket}
%\usepackage[enable]{todonotes}
\hypersetup{colorlinks=True,urlcolor=blue,linkcolor=blue,citecolor=blue,filecolor=black}
\usepackage[caption=false]{subfig}
% Extra commands
%\usepackage{lmodern}
%\usepackage[T1]{fontenc}
%\setcounter{secnumdepth}{3}
%\usepackage{color}
%\usepackage{float}
%\usepackage{mathtools}
%\usepackage{amssymb}
%\usepackage{breakurl}

%\usepackage{pslatex}  % Change font style.
%\renewcommand{\arraystretch}{1.5}  % Widens rows in Tables
%\setlength{\extrarowheight}{2pt}   % Widens rows in Tables

\makeatletter

%%%%%%%%%%%%%%%%%%%%%%%%%%%%%% LyX specific LaTeX commands.
%% Because html converters don't know tabularnewline
\providecommand{\tabularnewline}{\\}
%% A simple dot to overcome graphicx limitations
\newcommand{\lyxdot}{.}

\makeatother

\begin{document}

%Title of paper
\title{Shell-model project for Nuclear Talent course}

% repeat the \author .. \affiliation  etc. as needed
% \email, \thanks, \homepage, \altaffiliation all apply to the current
% author. Explanatory text should go in the []'s, actual e-mail
% address or url should go in the {}'s for \email and \homepage.
% Please use the appropriate macro foreach each type of information

% \affiliation command applies to all authors since the last
% \affiliation command. The \affiliation command should follow the
% other information
% \affiliation can be followed by \email, \homepage, \thanks as well.
\author{Group 1 - N. Gavrielov}
%\email[]{Your e-mail address}
%\homepage[]{Your web page}
%\thanks{}
%\altaffiliation{}
\affiliation{}

%Collaboration name if desired (requires use of superscriptaddress
%option in \documentclass). \noaffiliation is required (may also be
%used with the \author command).
%\collaboration can be followed by \email, \homepage, \thanks as well.
%\collaboration{}
%\noaffiliation

\date{\today}

\begin{abstract}
% insert abstract here
\end{abstract}

% insert suggested PACS numbers in braces on next line
\pacs{}
% insert suggested keywords - APS authors don't need to do this
%\keywords{}

%\maketitle must follow title, authors, abstract, \pacs, and \keywords
\maketitle
\section{Part 1c}
\subsection{Dimensionality}
The Hamiltonian matrix is calculated for $N$ single-particle-states (sps) and $n$ particles. Generally, (e.g. allowing pairs breaking) we have ${N \choose n}=\frac{N!}{n!(N-n)!}$ possibilities for ordering the particles. 

If we look at $k$-particle states (e.g. pairs, 2-particle states) we will have ${N \choose k}$ possibilities for ordering (e.g. for $N=8,k=4$)

levels that should be filled by $n/2$ pairs (where $N$ and $n$ are both even integers), which yields $\frac{(N/2)!}{[n/2]![(N-n)/2]!}$ possibilities for ordering the pairs. Working in the pairs representation, this will generate a $\frac{(N/2)!}{[n/2]![(N-n)/2]!}\times\frac{(N/2)!}{[n/2]![(N-n)/2]!}$ Hamiltonian matrix.

As an example, for 8-sps and 4-particles in the pairs representation, we have a dimensionality of ${4 \choose 2}=6$, whch yields a $6\times6$ Hamiltonian matrix.

\subsection{Calculating the states}
For a doubly-degenerate level with pair-particles (spin $1/2$ for each and total spin 0), there are 6 possible orderings written as

\begin{subequations}
\begin{align}
\ket{1,2,3,4} \leftrightarrow &\ket{1,2},\\
\ket{1,2,5,6} \leftrightarrow &\ket{1,3},\\
\ket{1,2,7,8} \leftrightarrow &\ket{1,4},\\
\ket{3,4,5,6} \leftrightarrow &\ket{2,3},\\
\ket{3,4,7,8} \leftrightarrow &\ket{2,4},\\
\ket{5,6,7,8} \leftrightarrow &\ket{3,4}.
\end{align}
\end{subequations}

The right column represents two pairs in two different levels (pair states), e.g. $\ket{1,2}$ represents a pair in level $P=1$ and a pair in level $P=2$. The left column represents the positions of all four particles (single particle states), e.g. $\ket{1,2,3,4}$ represents one particle in state number 1, one particle in state number 2, etc.

\subsection{Hamiltonian Matrix}

The Hamiltonian has two parts, the unperturbed Hamiltonian

\begin{equation}
\hat H_0 = \xi \sum_{p,\sigma} (p-1) \hat a_{p,\sigma}^\dagger \hat a_{p,\sigma},
\label{eq:H_0}
\end{equation}

with $\xi=1$, and the interacting part

\begin{equation}
\hat V = -\frac{1}{2} g \sum_{p,q} \hat P_p^+ \hat P_q^- ,\label{eq:V}
\end{equation}

where $\hat P_p^+$ and $\hat P_p^-$ are the pair creation and annihilation operators
\begin{equation}
\hat P_p^+ = \hat a_{p,+}^\dagger \hat a_{p,-}^\dagger, \qquad
\hat P_p^- = \hat a_{p,-} \hat a_{p,+} .
\label{eq:P}
\end{equation}

The full Hamiltonian then reads $\hat H = \hat H_0 + \hat V$. 

Next we calculate the Hamiltonian matrix. For the unperturbed part, we have

\begin{align}
\hat H_0 = \sum_{\sigma} \left[ \hat a_{2,\sigma}^\dagger \hat a_{2,\sigma} + 2 \hat a_{3,\sigma}^\dagger \hat a_{3,\sigma} + 3 \hat a_{4,\sigma}^\dagger \hat a_{4,\sigma}  \right]
\end{align}

\end{document}


